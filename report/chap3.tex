\chapter{Tables} \label{tables}


\section{Small tables and colour}
Tables can be used in a \texttt{table} environment and placed in on a well-suited place on the page. Little tables can also be used to format text without a \texttt{table} environment. This can be useful if you want to format text or graphics with specific alignments without making it a \textit{thing}. Here is an example without the \texttt{table} environment:
    
    
    \begin{tabular}{cc}
        first cell  & second  \\
        third       & fourth
    \end{tabular}
    
    A sentence and \textbf{}

    
    
    \begin{tabular}{lllll} %five columns, all left-aligned
    \toprule %Makes a horizonal line
        & Origin & Destination & Drive time & Drive distance \\
    \toprule %note the visual difference between \toprule and, below, \midrule
        \textbf{Input} & -37.7840 & RMIT University & & \\
    \midrule
        \textbf{Output} & 42 Mark St & 124 La Trobe St & 13 mins & 3.9 km \\
    \bottomrule
\end{tabular}

% Table generated by Excel2LaTeX from sheet 'Sheet1'
\begin{tabular}{p{10cm}ll}
\multicolumn{1}{r}{1} & \multicolumn{1}{r}{2} & \multicolumn{1}{r}{3} \\
a     & b     & c \\
\end{tabular}%

% Table generated by Excel2LaTeX from sheet 'Sheet1'
\begin{tabular}{lll}
\rowcolor[rgb]{ .388,  .745,  .482} \multicolumn{1}{r}{1} & \multicolumn{1}{r}{\cellcolor[rgb]{ 1,  .922,  .518}2} & \multicolumn{1}{r}{\cellcolor[rgb]{ .973,  .412,  .42}3} \\
a     & b     & c \\
\end{tabular}%




These tables can also be used for more subtle alignment purposes. For example, we can structure some text in a table within a column and add colour by using just the  \texttt{\textbackslash begin\{tabular\}} environment (and see \Vref{tbl:heatmap} for greater use of colour in tables).

    \begin{center} %an environment to horizontally center whatever is in it
    \begin{tabular}{ccc} %beginning the tabular environment
    
        \multicolumn{3}{c}{\cellcolor{Color5} \textcolor{white}{the variable}}  \\
        \multicolumn{3}{c}{\texttt{\small{\textcolor{Color5}{prior\_postgrad==21993}}}}  \\
        & \\ %empty line to add some space; adding empty space can also be done in alternative ways.
        \multicolumn{3}{c}{\cellcolor{Color4} \textcolor{white}{is split into}}  \\
        \texttt{\small{\textcolor{Color4}{prior\_postgrad\_cd==2}}}\quad & \& &  \texttt{\small{\textcolor{Color4}{prior\_postgrad\_yr==1993}}} \qquad \\
    \end{tabular}        
    \end{center}
    
While within-column tables are seldom used in Grattan work, they're good to keep in mind if you're having difficulty formatting something. 

But the most common way column-width tables will be used is within the \texttt{\textbackslash begin\{table\}} environment, like \Vref{tbl:distance}. This provides a caption and places the table in a responsible spot on the page (and here we add some colour with \texttt{\textbackslash cellcolor}). Also, note that we have made the `Drive distance' text wrap at a defined width. This is done using \texttt{p\{\}} instead of \texttt{l} in the column commands.

\begin{table}
    \caption{Example distance calculation using Google Maps}\label{tbl:distance} 
    \begin{tabular}{lp{1.9cm}p{2.9cm}p{1.7cm}p{1.5cm}} %See that we're defining the widths of the table using these commands. This bit of code says: we want the first column left-aligned (and make it as wide as the longest text), then the second column to be left-aligned and strictly 1.9cm wide (which invokes text-wrapping for text longer than 1.9cm), then the third column to be left-aligned and strictly 2.9cm wide, and so on. 
    \toprule
        & Origin & Destination & Drive time & Drive distance \\
    \toprule
        \textbf{Input} & -37.7840 & RMIT University & \cellcolor{Color1} & \cellcolor{Color1} \\
    \midrule
        \textbf{Output} & 42 Mark St & 124 La Trobe St & 13 mins & 3.9 km \\
    \bottomrule
\end{tabular}
\end{table}



\section{Full- and multi-page tables}

Multi-page tables can be generated using the \texttt{\textbackslash longtable} package.\footnote{This requires the \texttt{\textbackslash onecolumn} command, detailed in the code, which removes the two-column structure of Grattan documents. It therefore requires a \texttt{\textbackslash twocolumn} command at the end of the table.} Long-tables allow four additional commands before the content of the table that dictate headers and footers of the table: 

\vspace{-10pt} %reducing the space between the end of the paragraph and the table by 10pt
    \begin{center}
    \begin{tabular}{l|l}  %note the vertical line in the table added with '|' (the symbol above the backslash on a keyboard; apparently called a 'pipe') between the columns
        \texttt{\textbackslash endfirsthead} & The heading of the \textbf{first} page of the table.  \\
        \texttt{\textbackslash endhead} & The heading of all following pages of the table. \\
        \texttt{\textbackslash endfoot} & The footer of the table on pages before the last.  \\
        \texttt{\textbackslash endlastfoot} & The footer of the table on the \textbf{last} page. \\
    \end{tabular}
    \end{center}
\vspace{-10pt} %similarly, reducing the space between the end of the table and the next paragraph by 10pt    

A common use of this structure is to have  \textit{\footnotesize -- continued from previous page} on the top of each page after the first (\texttt{\textbackslash endhead}), and \textit{\footnotesize Continued on next page} on all pages of the table except the last (\texttt{\textbackslash endfoot}). 

There are \textit{hopefully-}helpful comments in each key line in the code of the \texttt{\textbackslash longtable} examples:\\
\vspace{10pt} %adding some vertical space 

\begin{center}
\begin{tabular}{l|l} \centering
    Two-page table & \Vref{tbl:surveyquestions} \\
    Very long table & \Vref{tbl:regressionresults} \\
    Full-page heatmap & \Vref{tbl:heatmap} \\
\end{tabular}
\end{center}



%% To view the code for the examples below, navigate your way to the tables folder on the left-hand side <- and select the one you want.
\input{tables/tbl_surveyquestions.tex}
% \input{tables/tbl_regressionresults.tex}
\input{tables/tbl_heatmap.tex}
